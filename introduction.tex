\section{Introduction}
\label{sec:introduction}

Wikipedia is the largest online encyclopaedia containing over 5 million pages of content. It is one the most popular websites on the Internet. Wikipedia has a diverse collection of articles from many different topics and is constantly being updated. Although Wikipedia started out as an open platform where anyone could create and articles, this lead to many factual errors and biased articles. Wikipedia started to incorporate elements hierarchy gradually over time. In the English version of Wikipedia all editors need to have a registered account, and pages that are controversial and of a sensitive nature are protected by administrators.

Administrators are editors who are given access to tools such as blocking and unblocking other users, deleting and undeleting pages, protecting and renaming pages etc. Any user can \textbf{Request for Adminship} (RfA) in which the Wikipedia community participates. The RfA spans over seven days, during which any editor can comment and discuss the candidate. In their online discussions, editors scrutinize the candidate's contributions and credentials as well their conduct in the online discussion and overall experience. They can then state either their support for or opposition to the candidate along with comments. At the end of seven days a Bureaucrat (an editor higher up in the hierarchy) decides on the "consensus" of the election and declare the outcome. Consensus is not a direct majority voting scheme, and the final call rests with the Bureaucrat.

The RfA is a very intense and selective process; there are only 1400 total administrators, of whom only 500 are currently active\footnote{all data as of March 2020 for English version Wikipedia}. This is out of 38 million registered editors with only around 130\,000 regular contributors. This small group of active administrators and editors are responsible for creating and maintaining all articles on Wikipedia.

Therefore, the RfA process can give us valuable insight into the dynamics of social interactions and elections in an online platform. In this paper we will first discuss the existing work on studying the RfA elections and other such similar online processes. Next we provide an overview of the data collected from Wikipedia and how it is used in this paper. We then present our main contribution, the use of a \textit{Viscous Democracy} to model the RfA election process. We discuss the results and possible extensions of this framework to other online elections systems.  
