% Elections are central in most democratic societies. 


% This is also the case with Wikipedia, the largest online encyclopedia hosted on the Internet. Wikipedia started out as an open society where anyone could create new Wikipedia articles and edited by any reader, even those who did not have a Wikipedia account. This initial system soon changed once the popularity of the encyclopedia started to grow. Many versions of Wikipedia, most notably the English Wikipedia started to only allow registered users to make edits on Wikipedia articles. The initial lawlessness of Wikipedia soon evolved to incorporate democratic principles and hierarchy. 

% Administrators or more commonly referred to as \textit{admins} are a group of Wikipedia editors who have been granted certain privileges and tools such as deletion of articles, prevent editing of certain pages, protect articles from vandalism etc. \cite{ViscousDemocracy}

\begin{itemize}
    \item how wikipedia is a large encyclopedia
    \item maintained by a small group of Administrators
    \item They undergo an election like process of RfAs
    \item How this is an important online social election framework
    \item How it has been studied in previous works 
    \item What we aim to do by using a social network and theories of democracy
\end{itemize}

Wikipedia is the largest online encyclopedia containing over 5 million pages of content. It is one the most popular websites on the Internet. Wikipedia has a diverse collection of articles from many different topics and is constantly being updated. Although Wikipedia started out as an open platform where anyone could create and articles, this lead to many factual errors and biased articles. Wikipedia started to incorporate elements hierarchy gradually over time. In the English version of Wikipedia all editors to have a registered account and pages that are controversial and of a sensitive nature are protected by administrators.
\smallskip

Although there are over 38 million registered editors\footnote{From here all information is for the English version of Wikipedia} only around 130 thousand are regular contributors to articles and the discussion forums. 


% For a Wikipedia member to rise to the level of an administrator, they or a nominator need to submit a request for Adminship (RfA). In this election any Wikipedia may cast a vote in support, opposition or neutral towards the candidate.
% This provides us with a very rich dataset on voting patterns withing Wikipedia members. The goal of this project is to collect data of such votes as well as data regarding individual Wikipedia member's contributions.
% This allows us to create and analyze networks of interactions among all Wikipedia users.
