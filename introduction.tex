% \begin{itemize}
%     \item how wikipedia is a large encyclopedia
%     \item maintained by a small group of Administrators
%     \item They undergo an election like process of RfAs
%     \item How this is an important online social election framework
%     \item How it has been studied in previous works 
%     \item What we aim to do by using a social network and theories of democracy
% \end{itemize}
\section{Introduction}
\label{sec:introduction}

Wikipedia is the largest online encyclopedia containing over 5 million pages of content. It is one the most popular websites on the Internet. Wikipedia has a diverse collection of articles from many different topics and is constantly being updated. Although Wikipedia started out as an open platform where anyone could create and articles, this lead to many factual errors and biased articles. Wikipedia started to incorporate elements hierarchy gradually over time. In the English version of Wikipedia all editors need to have a registered account and pages that are controversial and of a sensitive nature are protected by administrators.
\smallskip

Administrators are editors who are given access to tools such as blocking and unblocking other users, deleting and undeleting pages, protecting and renaming pages etc. Any user can \textbf{Request for Adminship}(RfA) in which the Wikipedia community participates. The RfA spans over seven days, during which any editor can comment and discuss the candidate. Editors scrutinize the candidate's contributions and credentials as well their conduct in the online discussion and overall experience. They can then state either their support or opposition to the candidate along with comments. At the end of seven days a Bureaucrat (an editor higher up in the hierarchy) decides on the consensus of the election and declare the outcome. Consensus is not a direct majority voting scheme and the final call rests with the Bureaucrat.
\smallskip

The RfA is a very intense and selective process, there are only 1400 total administrators of which only 500 are currently active\footnote{all data as of March 2020 for English version Wikipedia}. This is out of 38 million registered editors with only around 130 thousand are regular contributors. This small group of active administrators and editors are responsible for creating and maintaining all articles on Wikipedia.
\smallskip

Therefore the RfA process can give us valuable insight into the dynamics of social interactions and elections in an online platform. In this paper we will first discuss the existing work on studying the RfA elections and other such similar online processes. Next we provide an overview of the data collected and used from Wikipedia in this paper. We then present our main contribution, the use of a \textit{Viscous Democracy} to model the RfA election process. We discuss the results and possible extensions of this framework to other online elections systems.  


% For a Wikipedia member to rise to the level of an administrator, they or a nominator need to submit a request for Adminship (RfA). In this election any Wikipedia may cast a vote in support, opposition or neutral towards the candidate.
% This provides us with a very rich dataset on voting patterns withing Wikipedia members. The goal of this project is to collect data of such votes as well as data regarding individual Wikipedia member's contributions.
% This allows us to create and analyze networks of interactions among all Wikipedia users.
