\section{Viscous Democracy Model}
\label{sec:model}
In this section we explain the concept of viscous democracy \cite{ViscousDemocracy} and it's relevance in decision making on online platforms. We then go on to explain how we can use the concept of viscous democracy model to create a model to predict the results of Wikipedia RfA elections

\subsection{Concept}
There are three main forms of democracy namely \textbf{direct democracy}, \textbf{representative democracy}, and \textbf{liquid democracy}.

Direct democracy is when all the people are involved directly in deciding on any policy. it is the purest form of democracy and can only function well in relatively small social groups, as when the number of people cross a certain limit it is logistically infeasible to include everyone's opinion. Representative democracy (or indirect democracy) when people choose a representative who will carry out policy decisions in their favour. This system is widely used by most major countries as well social groups. The main drawback of this system of democracy is that the representatives are not obligated to fulfil promises to the society and once in power can further their self interest. The representatives also stay in power for a period of time within which the public's stance or view on policies might change.

Liquid democracy is in-between direct and representative democracy. It can be called \textbf{delegative democracy} as people have the chance to delegate their vote to a \textit{proxy} or choose to vote directly. This way the proxy's vote is augmented with all the delegations one receives and can also be transitive in that the proxy can again delegate to another individual. This style of proxy voting works due to \textbf{strong transitive delegation}. This means that I am more likely to trust my friend's friend to have similar interest as my own. There is a lot of theoretical and practical research ongoing in this field \cite{kahng2018liquid,hardt2015google}. The liquid democracy model is not directly suitable for online settings. This is because the social ties online are weak and therefore the transitivity of delegation is weak. It just means that you are less likely to trust your Facebook friend's other Facebook friend. 

There is a \textit{reluctance} to delegate in online communities. Hence the weight of the vote attenuates as it is delegated further down a chain. This can be visualized as the vote that is delegated as viscous and therefore its strength reduces with each additional delegation. This is the main concept behind \textbf{viscous democracy}. Every voting model requires a \emph{ballot} and a \emph{tally} which we will now describe.

\subsection{Ballot}
A ballot is how a voter expresses their preferences. In the examples of direct and indirect democracy the ballot is usually cast in the form of a vote, either directly for a policy or indirectly for a candidate. The form of voting can be a \textbf{one person, one vote} system where the voter indicates one option. The other is \textbf{ranking based voting} such as Instant-runoff voting (IRV)\footnote{\url{https://en.wikipedia.org/wiki/Instant-runoff_voting}} or Single Transferable Vote (STV)\footnote{\url{https://en.wikipedia.org/wiki/Single_transferable_vote}} where the voter ranks the options in the order of preference. There can also be ballot systems where you rate the options with a score.

In the case of delegative voting systems such as liquid or viscous democracy the choices are to either vote directly or choose to delegate to another person. If we restrict ourselves to only a single vote model then we can consider the ballot as making a \textit{delegation graph} \cite{ViscousDemocracy}. This concept is illustrated in Figure~\ref{fig:viscous-dem}. We assume that we have an underlying undirected social graph. Every node is a person and each edge indicates a connection. This can be like the friends network in Facebook or contacts networks in LinkedIn etc. Then the delegation graph is built upon the same nodes of the social graph where each node can either vote directly, leading to self loops or choose to delegate to one of their neighbours. Therefore if we assume there is some \textit{delegation rule} that each node follows then the delegation graph is induced from the social graph by applying that rule. 

\subsection{Tally}
\label{sec:tally}
The tally is the algorithm that decides the final outcome. In direct or indirect democracies the method of tally might be \textbf{first-past-the-post} an example of plurality based tallying or a proportional system where seats or power is awarded to the proportion of the votes received. 
 
The tally proposed by Boldi et al.\ is a scored based tally where each node in the network receives a score that is calculated based on the delegation graph \cite{ViscousDemocracy}. This score can then be used along with any requirement such as choosing a committee or one policy out of many alternatives. The method is called \textit{transitive proxy voting with exponential damping}, where a dampening factor $\alpha$ controls the amount of reluctance in the transitive delegation of votes. They proposed using \textbf{Katz's Centrality index} on the delegation graph. Therefore the score for a node $i$ is defined as 
\[\text{Score}_{i}  = \sum_{p \in \text{Path}(-,i)} \alpha^{|p|}\]
Here, $\text{Path}(-,i)$ is set of all delegation paths ending in node $i$ and $|p|$ is the length of the path \cite{ViscousDemocracy}. This step is shown in Figure~\ref{fig:viscous-dem}, where the \textit{Node Importance Graph} is the delegation graph where each node's size is proportional to the score that is obtained from Katz's Centrality with $\alpha=0.5$. Here we see that the nodes where the delegation chain ends have a larger node score as the votes are transitively transferred.

The parameter $\alpha\in (0,1)$ acts as the \textit{delegation factor} and $1-\alpha$ is the \textit{viscosity}. As $\alpha$ approaches $0$ the strength of votes delegated tend to $0$ hence direct votes have more weight. Therefore the more viscous the system is the more analogous it is to direct democracy or majority voting. As $\alpha$ approaches $1$ almost all the weight of the vote is transferred to the delegate and the node scores correspond to the size of the subtree that they are a part of and the vote and the system is more analogous to liquid democracy \cite{ViscousDemocracy}.

\subsection{Model} 
Now utilizing all the concepts behind \textit{viscous democracy} we propose a model to predict the Wikipedia RfA elections. This election model also consists of the two parts, namely ballot and tally. 

We can utilize the interactions between users on Wikipedia to create the underlying \textbf{social graph} that is required. Once we have a social graph we can use the many features from the user's contributions to implement a \textbf{delegation rule}. Applying the delegation rule upon the social graph we obtain the delegation graph. Then choosing an appropriate value of $\alpha$ and from Katz's centrality on the delegation graph we obtain scores for each node which corresponds to a user. We can then use the scores to tally votes in an election to predict the result using two approaches.

As we saw in Section~\ref{sec:influential-voters} a small core of voters are important in the result of an election. We can determine these influential voters either on a local scale or a global scale. On a local scale we can choose $k$ most influential voters in an election and then tally their votes to predict the election. On a global scale we can choose the top $k$ influential voters overall and then only tally their votes in a particular election to find the result. Both these approaches aim to use a smaller set of voters to effectively predict the result of an election.

\begin{figure*}[!ht]
    \centering
    \begin{tikzpicture}[remember picture,every
    node/.style={draw,circle,fill=blue,minimum width=4pt}]
        \node (C1) {};
        \node[below=0.7cm of C1]  (C2){};
        \node[right=0.8cm of C1] (C3){};
        \node[right=0.8cm of C2] (C4){};
        \node[below=0.7cm of C4]  (C5){};
        \node[right=0.8cm of C4]  (C6){};
        \draw (C4) -- (C5) -- (C6) -- (C2);
        \draw (C3) -- (C1) -- (C2) -- (C4) -- (C1);
        \draw (C3) -- (C6); 
        \node[draw=none,fill=none,rectangle,above=0.5cm of C1,xshift=-0.5cm,anchor=south west]{Social Graph };
        \coordinate[below=2.5cm of C1] (X);
        \draw[white](X)--++(1cm,0);
        \path (current bounding box.north east) -- 
        (current bounding box.south east) coordinate[midway] (2BL);
    \end{tikzpicture}
    \hspace*{2.5cm}%
    \begin{tikzpicture}[remember picture,every
    node/.style={draw,circle,fill=red,minimum width=4pt}]
        \node (C1) {};
        \node[below=0.7cm of C1]  (C2){};
        \node[right=0.8cm of C1] (C3){};
        \node[right=0.8cm of C2] (C4){};
        \node[below=0.7cm of C4]  (C5){};
        \node[right=0.8cm of C4]  (C6){};
        \draw[-latex] (C6) -- (C3);
        \draw[-latex] (C1) -- (C2);
        \draw[-latex] (C5) -- (C4);
        \draw[-latex] (C2) -- (C4);
        \path (C4) edge [loop above] (C4); 
        \path (C3) edge [loop left] (C3); 
        \node[draw=none,fill=none,rectangle,above=0.5cm of C1,xshift=-0.5cm,anchor=south
        west]{Delegation Graph};
        \coordinate[below=2.5cm of C1] (X);
        \draw[white](X)--++(1cm,0);
        \path (current bounding box.north west) -- 
        (current bounding box.south west) coordinate[midway] (2BR);
        \path (current bounding box.south west) -- 
        (current bounding box.south east) coordinate[midway] (2BB);
        \path (current bounding box.south east) -- 
        (current bounding box.north east) coordinate[midway] (2FM);
    \end{tikzpicture}
    \hspace*{3cm}%
    \begin{tikzpicture}[remember picture,every
        node/.style={draw,circle,fill=green,minimum size=1pt}]
        \node[scale=1] (C1) {};
        \node[below=0.7cm of C1,scale=1.75]  (C2){};
        \node[right=0.8cm of C1,scale=1.75] (C3){};
        \node[right=0.8cm of C2,scale=3.5] (C4){};
        \node[below right=0.8cm of C4,scale=1]  (C5){};
        \node[right=0.8cm of C4,scale=1]  (C6){};
        \draw[-latex] (C6) -- (C3);
        \draw[-latex] (C1) -- (C2);
        \draw[-latex] (C5) -- (C4);
        \draw[-latex] (C2) -- (C4);
        \path (C4) edge [loop below] (C4);
        \path (C3) edge [loop left] (C3); 
        \node[draw=none,fill=none,rectangle,above=0.5cm of C1,xshift=-0.5cm,anchor=south
        west]{Node Importance Graph};
        \coordinate[below=2.5cm of C1] (X);
        \draw[white](X)--++(1cm,0);
        \path (current bounding box.north west) -- 
        (current bounding box.south west) coordinate[midway] (3BU);
        \end{tikzpicture}
        \tikz[overlay,remember picture]{\draw[-latex,thick] (2BL) -- (2BL-|2BR)
        node[midway,above,text width=2.5cm]{Delegation Rule};} 
        \tikz[overlay,remember picture]{\draw[-latex,thick] (2FM) -- (3BU)
        node[midway,above,sloped,text width=2.5cm]{Katz Centrality $\alpha=0.5$};} 
        
        \caption{Viscous Democracy Model}
        \label{fig:viscous-dem}
    \end{figure*}