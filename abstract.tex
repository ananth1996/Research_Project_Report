\begin{abstract}
\label{sec:abstract}
The election process of administrators in the English version of Wikipedia is called a Request for Adminship (RfA). They provide us with s a unique insight into the voting behaviour of people in an online platform. We set out to find if a small group of voters can have an influence in the outcome of a RfA. Firstly, we discuss the process of collecting the election and voter data from Wikipedia and identifying a core-set of influential voters. Next, we explain the concept of viscous democracy; a proxy voting framework and its suitability for online elections where people are reluctant to trust others with their vote. We propose a model to compute a score for each voter using viscous democracy. Finally, we describe a local and a global approach to tally the votes in a RfA using only the most influential voters as determined by their scores. Our experiments show that the proposed model achieves a better predictive accuracy over the baseline of tallying all the votes in a RfA. Furthermore, we show that the votes in Wikipedia RfAs are of a viscous nature and how the viscosity parameter of the model affects the overall accuracy.
\end{abstract}
