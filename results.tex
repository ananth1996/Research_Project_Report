\begin{figure*}[t]
    \centering
    \includegraphics[width=\linewidth]{images/alpha_local.pdf}
    \caption{Effect of $\alpha$ for the \localv model with different delegation rules and values of $k$}
    \label{fig:local-alpha}
\end{figure*}
\begin{figure*}[t]
    \centering
    \includegraphics[width=\linewidth]{images/alpha_global.pdf}
    \caption{Effect of $\alpha$ for the \globalv model with different delegation rules and values of $k$}
    \label{fig:global-alpha}
\end{figure*}

\section{Results and Discussion}

\label{sec:results}
We presents the results of the viscous democracy model and analyze the effects of the various parameters on the accuracy of the model.
\smallskip


The main metric we will be using to compare the quality of the Viscous Model is \textbf{accuracy}. The baseline that we will be using to measure the model is a simple tally of all the votes in an RfA election. This compares the most directly with our models as we increase the value of $k$. As we discussed before generally Wikipedia RfA results are positive if there is at least $65\%$ supporting votes. Therefore we filter out elections that have fewer than 10 votes to get a better view of the predictive quality of the model.The baseline model of just tallying the votes gives an accuracy of $\mathbf{82\%}$. 
\smallskip

There are many parameters that have an effect on the predictive accuracy of the model. This means that the effective size of search space is quite large especially as parameters like $\alpha$ have a continuos domain. To narrow the search space as well gather a better understanding of the effects each parameter has on the model we fix all others and take a closer look in the following subsections. Finally we discuss the results of the grid search over the pruned parameter space to find the best performing model.

\begin{figure*}[t]
    \centering
    \includegraphics[width=\linewidth]{images/k_global.pdf}
    \caption{Effect of $k$ for the \globalv model with different delegation rules and values of $\alpha$}
    \label{fig:global-k}
\end{figure*}
\begin{figure*}[t]
    \centering
    \includegraphics[width=\linewidth]{images/k_local.pdf}
    \caption{Effect of $k$ for the \localv model with different delegation rules and values of $\alpha$}
    \label{fig:local-k}
\end{figure*}
\subsection{Effect of $\alpha$}
To study the effects of the delegation parameter $\alpha$ we need to fix $k$, the delegation rule and the tally method. As the range of $k$ depends on the tally method, we split these results firstly as either using the \globalv or the \localv model.
\smallskip

In the \globalv model, $k \in [1,13000]$ and therefore we pick 5 values of $k$ and plot the accuracy of the model independently using each delegation rule. We see the results in Figure~\ref{fig:global-alpha}. There are some general trends that we can see across all the delegation rules. For small valued of $k$ the quality of predictions are low and as a result the overall accuracy is poor and usually below $50\%$. This is as expected because in the global tally scheme the top $100$ important users might not vote in all elections hence the model has very few votes to tally in each election. For large values of $k$ such as $8000$, we have close to two-thirds of the unique voters and then the value of $\alpha$ is irrelevant as the ranking of the top $8000$ users changes very little. Almost all votes in an election are used to calculate the tally hence the accuracy is constant close to the baseline $80\%$. For values of $k$ in the middle, we see that as $\alpha$ increases there is a gradual drop in the accuracy. This is clearly seen in the step like decrease for all delegation rules in Figure~\ref{fig:global-alpha}. This indicates that the \globalv model is more viscous in nature and works better for smaller values of $\alpha$.
\smallskip

When we consider the \localv model we see that the value of $k$ is bounded by the number of votes in a particular election. Theoretically we can have a $k$ more than the number of votes in an election, in which case we just choose all the votes to get the tally. The average number of votes in a RfA election is $49$ and we chose the following 5 values $k \in \{5,10,20,40,80\}$ to analyze. We again see that for $k=5$ is accuracy is low and for $k=80$ the accuracy is close to the baseline and stable. The reasoning is similar to the \globalv model, for small $k$ not enough votes and for large $k$ the $\alpha$ does not change ranking enough to affect the top $k$ users. The interesting trend is that unlike the \globalv model when we increase $k$ to $20$ we see that the \localv model actually performs much better than larger values of $k$ as well as the baseline. This indicates that around the region $k=20$ there is additional gain performance in choosing only the important votes to tally. We also see the trend that higher values of $\alpha$ tend to decrease the accuracy, especially for the seniority rule as seen in Figure~\ref{fig:local-alpha}. A point to not is that for most values of $k$ and delegation rules the performance drops significantly at the point $\alpha=0.5$, this indicates there might be a significant change in the ranking of voters in the \localv model. Therefore the local model is also \textbf{more viscous than liquid}.

\subsection{Effect of $k$}
Similar to how we studied the effect of $\alpha$ above, now we take the two tally methods, delegation rules and fixed values of $\alpha \in \{0.2,0.4,0.6,0.8\}$ to understand the effect of the value of $k$. Though we have in a way analyzed this in the previous subsection, now we provide a more detailed look of the trends as well as the ranges to choose for $k$ when later performing the grid search. We will again analyze the \globalv and \localv model separately.
\smallskip

In Figure~\ref{fig:local-k} we see that the \globalv model has a knee around $k=2000$, this is where we see the performance of the model approaching the baseline. It also around this region that the effect of the delegation parameter $\alpha$ is most prominent. In line with the previous analysis, we see that the smaller values of $\alpha$ perform much better than the larger values of $\alpha$ in this region. We can also verify that for both small and large $k$ the change in $\alpha$ has no effect. \
\smallskip

We see a similar knee for the \localv model in Figure~\ref{fig:local-k}, this time just before $k=20$. In this knee region there are some differences compared to the \globalv plots. We see the effect of $\alpha$ is not as pronounced in this region. We also see that there are distinctive spikes that lead to a bump before plateauing back to the baseline as $k$ increases. These spikes confirm the behaviour that we discussed previously in Figure~\ref{fig:local-alpha}, around this region the \localv model has slightly better performance than the baseline. This is again clearly evident in the $\alpha=0.4$ line for the seniority delegation rule in Figure~\ref{fig:local-k}.

\subsection{Grid Search}

\begin{table}
    \centering
    \begin{tabular}{@{\extracolsep{\fill}}llccc}
        \toprule
        Delegation Rule & Tally Method & k & $\alpha$ & Accuracy \\ \midrule
        \multirow{2}{*}{Seniority} & Global & 6 & 230  \\ 
        \cmidrule{2-5}
        & Local & 15 & 0.303& 0.832  \\
        \midrule
        \multirow{2}{*}{Edit Count} & Global & 6 & 230  \\
        \cmidrule{2-5}
        & Local & 5 & 195  \\
        \midrule
        \multirow{2}{*}{Rank} & Global & 6 & 230  \\
        \cmidrule{2-5}
        & Local & 5 & 195  \\
        \midrule
        \multirow{2}{*}{Reveresed Rank} & Global & 6 & 230  \\
        \cmidrule{2-5}
        & Local & 5 & 195  \\
        \bottomrule
        \end{tabular}
        \caption{The best combinations of parameter and the accuracy of the resulting model}
        \label{tab:grid-search}
\end{table}

The quality of predictions using local or global important editors.