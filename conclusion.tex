\section{Conclusions}
\label{sec:conclusion}
In this paper we explore the election process for administrator in Wikipedia. We consider the problem of election result prediction using a subset of votes cast. We proposed a novel approach to using the theory of viscous democracy to obtain a core set of influential users and predict elections by tallying their votes. This model allows us to understand the nature of election in Wikipedia and the degree of transitive trust present within the network of users. The experimental results show that our model is able to match and also outperform the baseline while only using a subset of the voters.

The small values of the delegation parameter show that the transitive trust in delegating votes is weak and that models with viscous votes perform better. We also see that delegation on the basis of seniority achieves the best performance and indicates that older Wikipedia members wield some influence in administrator elections.

In the future, we would like to create a method to infer the best delegation network and in turn the most optimal delegation rule. Another interesting approach would be to find a model that can explain the behaviour of a voter in a given election that provides a micro viewpoint of the election. This, when combined with the macro viewpoint provided by the viscous democracy model could provide a comprehensive understanding of election dynamics in Wikipedia.